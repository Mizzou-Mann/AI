% hw2.tex - Solution for homework 2 - Artificial Intelligence

% Chanmann Lim - September 2014

\documentclass[a4paper]{article}

\usepackage[margin=1 in]{geometry}
\usepackage{amsmath}
\usepackage[]{algorithm2e}

\everymath{\displaystyle}

\begin{document}
\title{CS 7750: Solutions to homework 2}
\author{Chanmann Lim}
\date{\today}
\maketitle

\section*{3.2}

\paragraph{a.} ~\\
\indent States: all possible location and facing direction of the robot on the maze's map \\
\indent Initial state: (centre, north) \\
\indent Actions: \{ turn north, turn east, turn south, turn west, move forward \} \\
\indent Transition model: returns new (location, facing) of the robot  \\
\indent Goal test: checks whether the location of the robot is out of the maze \\
\indent Path cost: each step costs 1, path cost = total number of step taken \\

\indent State space: $4 \times available \; squares \; on \; the \; maze$

\paragraph{b.} ~\\
\indent Actions: \{ turn north, turn east, turn south, turn west, move to intersection \} \\
\indent State space: $4 \times (four \; intersection \; squares) + 3 \times (three \; intersection \; squares) + 2 \times (two \; intersection \; squares) + (corridor \; without \; intersection \; squares)$

\paragraph{c.} ~\\
\indent Actions: \{ turn north, turn east, turn south, turn west, move until turning point \} \\
\indent State space: $4 \times the \; number \; of \; turing \; points$ \\

\indent Now, it is not necessary to keep track of the robot's orientation since the action of turning robot's will not happen before reaching the turning point on the maze.

\paragraph{d.}
There are some details from the real world have been abstracted to simplify the problem formulation such: \\
\indent 1. The road condition of the maze (without crater, sand or water which will increase the complexity in moving) \\
\indent 2. How the robot move (by wheel, leg or fly) is abstracted \\
\indent 3. Turning precision. We assume the robot will always in the middle of the corridor when turning so that it will not hit the side when moving

\section*{3.3}

\paragraph{a.} ~\\
\indent States: cities map with location of both friends \\
\indent Initial state: (city\_1, city\_2) \\
\indent Actions: move each friend to neighboring city \\
\indent Transition model: returns current cities of friend 1 and friend 2 after moving \\
\indent Goal test: checks whether if city\_1 and city\_2 is the same city \\
\indent Path cost: step cost = max(d1(i, j), d2(i, j)), path cost = total step cost \\

\paragraph{b.} $D(i, j)$ and $D(i, j)/2$ heuristic functions are admissible since both functions yield the result that is never over estimate the real distance, in other word the distance to the goal from n "$h^{*}(n)$" is always larger than the straight-line distance "$D(i, j)$" thus larger than "$D(i, j)/2$". However, there is no way to proof that $h^{*}(n)$ will be larger than heuristic function $h(n) = 2 \cdot D(i, j), thus it is not an admissible heuristic function$.

\paragraph{c.} Yes, if there are only 2 cities A and B (connected) and each friend is in one city, they will go in the opposite direction to see each other but will never meet.

\vspace{1 cm}

\paragraph{d.} Yes, suppose there are only 2 cities (connected) but one city has an extra circular route connected to itself, then one friend can go from $A -\textgreater A$ and other can go from $B -\textgreater A$ so that they can meet each other. In this scenario, one friend has to visit city A twice. 

\vspace{1 cm}

\section*{3.5}

\indent By counting the possibilities of placing one queen at a time on each column (or row) we can derive that there at least $n$, $(n-3)$, $(n-6)$, $(n-9)$ ... possibilities for column (or row) 1, 2, 3, 4 ... respectively when n is a large number. So the number of possibilities is
\begin{align*}
A = n \cdot (n-3) \cdot (n-6) \cdot (n-9) \ldots
\end{align*}
Let
\begin{align*}
B = (n-1) \cdot (n-4) \cdot (n-7) \ldots \\
C = (n-2) \cdot (n-5) \cdot (n-8) \ldots
\end{align*}

\begin{align*}
A \cdot B \cdot C &= n \cdot (n-1) \cdot (n-2) \cdot (n-3) \cdot (n-4) \cdot (n-5) \cdot (n-6) \cdot (n-7) \cdot (n-8) \cdot (n-9) \ldots \\
A \cdot B \cdot C &= n!
\end{align*}
Since A, B, and C should have the same term to n! and $A > B$ and $A > C$

\begin{align*}
A^{3} > n! \\
A > \sqrt[3]{n!}
\end{align*}
So that the state space has at least $\sqrt[3]{n!}$ states.

\vspace{1 cm}

\section*{3.6}

\paragraph{a.} ~\\
\indent States: regions with colors but no repeating color next to each others \\
\indent Initial state: regions without any color on the map \\
\indent Actions: fill any color to a region on the map such that the adjacent regions does not have the same color \\
\indent Transition model: returns the map with a color is filled in the specified region \\
\indent Goal test: checks whether all regions are colored and no two adjacent regions have the same color \\
\indent Path cost: each region coloring costs 1, path cost = total regions filled with color \\

\paragraph{b.} ~\\
\indent States: monkey position, crate 1 and crate 2 status(stacked?) \\
\indent Initial state: the monkey is on the ground and 2 crates are not stacked \\
\indent Actions: \{ stack, climb \} \\
\indent Transition model: returns the monkey position and 2 crates' status \\
\indent Goal test: checks whether if the monkey reaches the banana $height_{(monkey)} \geq height_{(banana)} $ \\
\indent Path cost: each action costs 1, path cost = the number of actions performed

\section*{3.15}

\paragraph{a.} ~\\

\vspace{4 cm}

\paragraph{b.} ~\\
\indent Breath first: [ 1, 2, 3, 4, 5, 6, 7, 8, 9, 10, 11 ] \\
\indent Depth-limit(3): [ 1, 2, 4, 8, 9, 5, 10, 11 ] \\
\indent Iterative deepening: [1, \{1, 2, 3\}, \{1, 2, 3, 4, 5, 6, 7\}, \{1, 2, 3, 4, 5, 6, 7, 8, 9, 10, 11\}]

\paragraph{c.} The bidirectional search will work well in this problem since the graph is a tree and there is only one predecessor of any states and the backward search can traverse directly to the root. \\

\indent Branching factor for forward direction (initial -\textgreater goal): 2 \\
\indent Branching factor for backward direction (goal -\textgreater initial): 1

\vspace{4 cm}

\paragraph{d.} Yes, the backward direction(goal -\textgreater initial)'s branching factor is 1 and this indicates that there will be no search required in that direction since there is only one route mapped from state 1 to any goal state in the backward direction.

\paragraph{e.} ~\\
\begin{algorithm}[H]
	\KwData{goal: the goal number(integer)}
	\KwResult{solution: the solution to the goal number}
	$i \longleftarrow goal$\;
	$solution \longleftarrow [1]$\;
	\While{ $(i/2) \neq 0$ }{
		$solution \longleftarrow INSERT(i, 1)$\;
		$i \longleftarrow i / 2$\;
	}
	\KwRet{$solution$}\;
\end{algorithm}

\end{document}